% Assume-se que \textual já foi feito

\chapter{Exemplo}
\label{ch:exemplo}

\index{bobagem} Primeiro parágrafo da seção com uma frase sem sentido que só serve para ocasionar uma quebra e de demonstrar a configuração de indentação da primeira linha. Essa frase está aqui pois parágrafos de uma linha são feios.

Resultado do uso de siglas:
\begin{itemize}
\item Sigla que nunca expande: \API;
\item Sigla normal, expande no primeiro uso: \DHT, mas não no segundo: \DHT;
\item Siglas com plurais automaticos: \APIs e \DHTs;
\item Forçando uma expansão (e no plural) \Glsfirstplural{DHT};
\item Usando uma sigla cujo comando é diferente da sigla: \WTC.
\end{itemize}

Resultado do glossário:
\begin{itemize}
\item Dois termos, \polling e \proxy;
\item Plural: \proxys.
\end{itemize}

Resultado de \mla|index|: primeiro um link normal \indexterm{tomate}, depois um capitalizado \indexTerm{tomate}.

\begin{defn}
  Exemplo de definição
\end{defn}

\begin{theorem}
  Exemplo de teorema
\end{theorem}

\begin{theoremproof}
  Exemplo de prova \qed
\end{theoremproof}

(Sub)enumerações e citações (verificar se OK com o idioma):
\begin{enumerate}
\item \cite{turing1937}:
  \begin{enumerate}
  \item \citeonline{turing1937}:
    \begin{enumerate}
    \item \citeonline{dijkstra1968};
    \end{enumerate}
  \end{enumerate}
\item \cite{turing1937,dijkstra1968};
\item \citeonline{turing1937,dijkstra1968}.
\end{enumerate}


\begin{listing}[tb]
\caption{Meta informações do presente documento.}
\label{lst:meta}
\begin{minted}[highlightlines={1,4-5}]{latex}
\titulo{Template \LaTeX{} para testes e dissertações do LAPESD/UFSC}
\autor{Omar Ravenhurst}
\data{1 de agosto de 2019} % ou \today
\tese % ou \dissertacao
\titulode{Doutor em Ciência da Computação}
\orientador{Prof. Dr. Ben Trovato}
\coorientador{Prof. Dr. Lars Thørväld}

\membrobanca{Prof. Valerie Béranger, Dr.}{Universidade Federal de Santa Catarina}
\membrobanca{Prof. Mordecai Malignatus, Dr.}{Universidade Federal de Santa Catarina}
\membrobanca{Prof. Huifen Chan, Dr.}{Universidade Federal de Santa Catarina}
\coordenador{Prof. Dr. Charles Palmer}
\end{minted}
\fonte{o autor.}
\end{listing}


Resultado de \mla|\autoref|s:
\begin{itemize}
\item \autoref{lst:meta};
\item \autoref{alg:algoritmo};
\item \autoref{fig:figura} tem subfiguras:
  \begin{itemize}
  \item \autoref{fig:svg}
  \item \autoref{fig:brasao}
  \end{itemize}
\item \autoref{tb:tabela};
\item \autoref{ch:exemplo};
\item \autoref{sec:frutas};
\item \autoref{sec:goiaba};
\item \autoref{sec:jabuticaba};
\item \autoref{sec:tomate}.
\end{itemize}


\begin{algorithm}
  \caption{Exemplo do ambiente \texttt{algorithimic}.}
  \label{alg:algoritmo}
  \begin{algorithmic}[1]
    \Procedure{Closure}{C, A}
      \State{$H \gets \emptyset$}\Comment{Direct cache}
      \For{$i \in [1, n]$}\Comment{Parallel, (dynamic,32) scheduling}
        \State{$H \gets H \cup \Call{DoImportantStuff}{i}$}
      \EndFor
    \EndProcedure
  \end{algorithmic}
  \fonte{o autor.}
\end{algorithm}

\begin{figure}[tb]
  \centering
  \caption{Exemplo de figura com duas subfiguras.}   
  \label{fig:figura}
  
  % Subfiguras são feitas usando as funcionalidades do memoir. Não
  % inclua outros pacotes, pois eles podem fazer o memoir dar ragequit
  % 
  % Há duas maneiras, a maneira limpinha (só no lapesd-thesis.cls) e a
  % maneira do memoir (aviso: \subtop não funciona direito).
  \subcaptionminipage[fig:svg]%
    {.49\linewidth}%
    {O Makefile compila SVGs em PDFs usando o inkscape}%
    {\includegraphics[width=.2\linewidth]{alphachannel.pdf}}%
  \hfill% 
  % o comando acima expande para o equivalente disso:
  \begin{minipage}[t]{.49\linewidth}%
    \centering
    \subcaption{Brasão da UFSC.\label{fig:brasao}}
    \includegraphics[width=.2\linewidth]{\jobname-logo.pdf}
  \end{minipage}

  \fonte{o autor.}
\end{figure}

\begin{table}[tb]
  \centering
  \caption{Exemplo de tabela e símbolos}
  \label{tb:tabela}
  \begin{tabular}{lccp{5cm}}
    \toprule
    Esquerda & Coluna 1    & \rotatebox{90}{90 graus}  & Parágrafo com \mla|p{5cm}|   \\
    \midrule
    $r_1$    & \cmk        &  \xmk                     & \circledi    \\
    $r_2$    &     \multicolumn{2}{c}{merged cell}     & \circledii   \\
    $r_3$    & \circlediii & \circlediv                & \circledv    \\
    $r_4$    & \circledvi  & \circledvii               & \circledviii \\
    $r_5$    & \circledix  &  x                        & y           \\
    \bottomrule 
  \end{tabular}
  \fonte{o autor.}
\end{table}

\section{Frutas}
\label{sec:frutas}
\lipsum[4]

\subsection{Goiaba}
\label{sec:goiaba}
\lipsum[4]

\subsubsection{Jabuticaba}
\label{sec:jabuticaba}
\lipsum[4]

\subsubsubsection{Tomate}
\label{sec:tomate}

\xindex{tomate} \lipsum[4]


%%% Local Variables:
%%% mode: latex
%%% TeX-master: "main"
%%% End:
