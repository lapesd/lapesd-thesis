% Assume-se que \pretextual já foi feito


\imprimircapa%
\imprimirfolhaderosto*
% Atenção! esse \protect é importante
\protect\incluirfichacatalografica{ficha.pdf}
\imprimirfolhadecertificacao


\begin{dedicatoria}
  Este trabalho é dedicado à wikipedia e ao stackoverflow. Essa frase está aqui apenas para testar o alinhamento e as margens
\end{dedicatoria}


\begin{agradecimentos}
  O presente trabalho foi realizado com apoio da Coordenação de Aperfeiçoamento de Pessoal de Nível Superior -- Brasil (CAPES) -- Código de Financiamento 001.
\end{agradecimentos}


\begin{epigrafe}
  For a number of years I have been familiar with the observation that the quality of programmers is a decreasing function of the density of go to statements in the programs they produce \\
  \cite{dijkstra1968}
\end{epigrafe}


\begin{resumo}[Resumo]
  Aqui deve ser inserido um resumo de 150 a 500 palavras (limitação de tamanho dada pela BU). A linguagem deve ser português e a hifenização já foi alterada. O resumo em português deve preceder o resumo em inglês, mesmo que o trabalho seja escrito em inglês. A BU também diz que deve ser usada a voz ativa e o discurso deve ser na 3ª pessoa. A estrutura do resumo pode ser similar a estrutura usada em artigos: Contexto -- Problema -- Estado da arte -- Solução proposta  -- Resultados.

  % Atenção! a BU exige separação através de ponto (.). Ela recomanda de 3 a 5 keywords
  \vspace{\baselineskip} 
  \textbf{Palavras-chave:} Palavra-chave. Ponto como separador. Bla.
\end{resumo}


\begin{resumo}[Resumo Estendido]
%%%%%%%%%%%%%%%%%%%%%%%%%%%%%%%%%%%%%%%%%%%%%%%%%%%%%%%%%%%%%%%%%%%%%%
% Atenção: normas e templates contraditórios!!!                    %%%
%%%%%%%%%%%%%%%%%%%%%%%%%%%%%%%%%%%%%%%%%%%%%%%%%%%%%%%%%%%%%%%%%%%%%%
% - Modelo da BU: https://repositorio.ufsc.br/handle/123456789/197458
% - A BU exige no **mínimo** 2 páginas e no **máximo** 5
% - Regimento do PPGCC, Art 40 Entende-se  por  resumo  estendido  um  documento  que  contenha  as  informações  mais  relevantes  de  cada  capítulo  da  tese  ou  da  dissertação.
% O mais seguro é ignorar o regimento e seguir a BU.
    % Atenção! A BU diz que o resumo **deve** conter as seções abaixo!
  \section*{Introdução} % Deve ser  subsection*, devido a formatação usada no modelo
  A hifenização é alterada para \texttt{brazil}, mesmo para documentos em inglês. Descrever brevemente esses itens exigidos pela BU. Como a RN 95/CUn/2017 é mais recente e impõe outras regras a revelia de regimentos e regulamentos, é mais sábio obedecê-la. Lembre que esse resumo estendido deve term entre 2 e 5 páginas.
  
  \lipsum[1]
  \section*{Objetivos} 
  \lipsum[21]
  \section*{Metodologia} 
  \lipsum[3]
  \section*{Resultados e Discussão} 
  \lipsum[4]
  \section*{Considerações Finais} 
  \lipsum[5]

  \vspace{\baselineskip}  % Atenção! manter igual ao resumo
  \textbf{Palavras-chave:} Palavra-chave. Outra Palavra-chave composta. Bla.
\end{resumo}


\begin{abstract}
  Enlish version of the plain ``resumo'' above. Hyphenization is automatically changed to english

  \vspace{\baselineskip} 
  \textbf{Keywords:} Keyword. Another Compound Keyword. Bla.
\end{abstract}

\listoffigures*  % O * evita que apareça no sumário
\listoftables*
\listoflistings*  
\listofalgorithms*

\listadesiglas*[5em]

\begin{listadesimbolos}
  $\gets$   & Atribuição \\
  $\exists$   & Quantificação existencial \\
  $\rightarrow$   & Implicação \\
  $\wedge$   & E lógico \\
  $\vee$   & Ou lógico \\
  $\neg$   & Negação lógica \\
  $\mapsto$   & Mapeia para \\
  $\sqsubseteq$   & Subclasse (em ontologias) \\
  $\subseteq$   & Subconjunto: $\forall x\;.\; x \in A \rightarrow x \in B$ \\
  $\langle\ldots\rangle$ & Tupla \\
  $\forall$   & Quantificação universal \\
  mmmmm & Nenhum sentido, apenas estou aqui para demonstrar a largura máxima dessas colunas. Ao abrir o ambiente \texttt{listadesimbolos}, pode-se fornecer um argumento opcional indicando a largura da coluna da esquerda (o default é de 5em): \texttt{\textbackslash{}begin\{listadesimbolos\}[2cm] .... \textbackslash{}end\{listadesimbolos\}} \\
\end{listadesimbolos}

\tableofcontents*%

%%% Local Variables:
%%% mode: latex
%%% TeX-master: "main"
%%% End:
